\documentclass[10pt,a4paper]{article}

% Packages
\usepackage{fancyhdr}           % For header and footer
\usepackage{multicol}           % Allows multicols in tables
\usepackage{tabularx}           % Intelligent column widths
\usepackage{tabulary}           % Used in header and footer
\usepackage{hhline}             % Border under tables
\usepackage{graphicx}           % For images
\usepackage{xcolor}             % For hex colours
%\usepackage[utf8x]{inputenc}    % For unicode character support
\usepackage[T1]{fontenc}        % Without this we get weird character replacements
\usepackage{colortbl}           % For coloured tables
\usepackage{setspace}           % For line height
\usepackage{lastpage}           % Needed for total page number
\usepackage{seqsplit}           % Splits long words.
%\usepackage{opensans}          % Can't make this work so far. Shame. Would be lovely.
\usepackage[normalem]{ulem}     % For underlining links
% Most of the following are not required for the majority
% of cheat sheets but are needed for some symbol support.
\usepackage{amsmath}            % Symbols
\usepackage{MnSymbol}           % Symbols
\usepackage{wasysym}            % Symbols
%\usepackage[english,german,french,spanish,italian]{babel}              % Languages

\newcommand\tab[1][5mm]{\hspace*{#1}}

% Lengths and widths
\addtolength{\textwidth}{6cm}
\addtolength{\textheight}{-1cm}
\addtolength{\hoffset}{-3cm}
\addtolength{\voffset}{-2cm}
\setlength{\tabcolsep}{0.2cm} % Space between columns
\setlength{\headsep}{-12pt} % Reduce space between header and content
\setlength{\headheight}{85pt} % If less, LaTeX automatically increases it
\renewcommand{\footrulewidth}{0pt} % Remove footer line
\renewcommand{\headrulewidth}{0pt} % Remove header line
\renewcommand{\seqinsert}{\ifmmode\allowbreak\else\-\fi} % Hyphens in seqsplit
% This two commands together give roughly
% the right line height in the tables
\renewcommand{\arraystretch}{1.3}
\onehalfspacing

% Commands
\newcommand{\SetRowColor}[1]{\noalign{\gdef\RowColorName{#1}}\rowcolor{\RowColorName}} % Shortcut for row colour
\newcommand{\mymulticolumn}[3]{\multicolumn{#1}{>{\columncolor{\RowColorName}}#2}{#3}} % For coloured multi-cols
\newcolumntype{x}[1]{>{\raggedright}p{#1}} % New column types for ragged-right paragraph columns
\newcommand{\tn}{\tabularnewline} % Required as custom column type in use

% Font and Colours
\definecolor{HeadBackground}{HTML}{333333}
\definecolor{FootBackground}{HTML}{666666}
\definecolor{TextColor}{HTML}{333333}
\definecolor{DarkBackground}{HTML}{0764A3}
\definecolor{LightBackground}{HTML}{EFF5F9}
\renewcommand{\familydefault}{\sfdefault}
\color{TextColor}

% Header and Footer
\pagestyle{fancy}
\fancyhead{} % Set header to blank
\fancyfoot{} % Set footer to blank


\begin{document}
\raggedright
\raggedcolumns

% Set font size to small. Switch to any value
% from this page to resize cheat sheet text:
% www.emerson.emory.edu/services/latex/latex_169.html
\footnotesize % Small font.

\begin{multicols*}{2}

\begin{tabularx}{8.4cm}{X}
\SetRowColor{DarkBackground}
\mymulticolumn{1}{x{8.4cm}}{\bf\textcolor{white}{Ein- und Ausgabe}}  \tn
% Row 0
\SetRowColor{LightBackground}
\mymulticolumn{1}{x{8.4cm}}{`var = int(input('prompt'))`} \tn 
% Row Count 1 (+ 1)
% Row 1
\SetRowColor{white}
\mymulticolumn{1}{x{8.4cm}}{`var = float(input('prompt'))`} \tn 
% Row Count 2 (+ 1)
% Row 2
\SetRowColor{LightBackground}
\mymulticolumn{1}{x{8.4cm}}{`print('prompt')`} \tn 
% Row Count 3 (+ 1)
% Row 3
\SetRowColor{white}
\mymulticolumn{1}{x{8.4cm}}{`print('text \%s text' \%(var))`} \tn 
% Row Count 4 (+ 1)
% Row 4
\SetRowColor{LightBackground}
\mymulticolumn{1}{x{8.4cm}}{`print('text \{\} text'.format(var))`} \tn 
% Row Count 5 (+ 1)
\hhline{>{\arrayrulecolor{DarkBackground}}-}
\SetRowColor{LightBackground}
\mymulticolumn{1}{x{8.4cm}}{`\%s` Platzhalter Textvariable, `\%d` Platzhalter Zahlvariable}  \tn 
\hhline{>{\arrayrulecolor{DarkBackground}}-}
\end{tabularx}
\par\addvspace{1.3em}

\begin{tabularx}{8.4cm}{p{1.08 cm} x{3.096 cm} p{1.08 cm} x{1.944 cm} }
\SetRowColor{DarkBackground}
\mymulticolumn{4}{x{8.4cm}}{\bf\textcolor{white}{Operatoren}}  \tn
% Row 0
\SetRowColor{LightBackground}
`x+y` & Addition & `x-y` & \seqsplit{Subtraktion} \tn 
% Row Count 2 (+ 2)
% Row 1
\SetRowColor{white}
`x*y` & Multiplikation & `x/y` & Division \tn 
% Row Count 3 (+ 1)
% Row 2
\SetRowColor{LightBackground}
`x\%y` & Modulo & \seqsplit{`x**y`} & x\textasciicircum{}y\textasciicircum{} \tn 
% Row Count 4 (+ 1)
% Row 3
\SetRowColor{white}
\seqsplit{`x//y`} & Division ohne Rest &  &  \tn 
% Row Count 6 (+ 2)
\hhline{>{\arrayrulecolor{DarkBackground}}----}
\end{tabularx}
\par\addvspace{1.3em}

\begin{tabularx}{8.4cm}{x{1.84 cm} x{6.16 cm} }
\SetRowColor{DarkBackground}
\mymulticolumn{2}{x{8.4cm}}{\bf\textcolor{white}{Datentypen}}  \tn
% Row 0
\SetRowColor{LightBackground}
Integer & -25, 23 \tn 
% Row Count 1 (+ 1)
% Row 1
\SetRowColor{white}
Float & -2.34, 65.3 \tn 
% Row Count 2 (+ 1)
% Row 2
\SetRowColor{LightBackground}
String & 'Hello', "World", """multiline""" \tn 
% Row Count 4 (+ 2)
% Row 3
\SetRowColor{white}
Boolean & True, False \tn 
% Row Count 5 (+ 1)
% Row 4
\SetRowColor{LightBackground}
List & {[}value, ...{]} \tn 
% Row Count 6 (+ 1)
% Row 5
\SetRowColor{white}
Tupel & (value, ...)\textasciicircum{}1\textasciicircum{} \tn 
% Row Count 7 (+ 1)
% Row 6
\SetRowColor{LightBackground}
\seqsplit{Dictionary} & \{key:value,...\} \tn 
% Row Count 9 (+ 2)
% Row 7
\SetRowColor{white}
Set & \{value, value,...\}\textasciicircum{}2\textasciicircum{} \tn 
% Row Count 10 (+ 1)
\hhline{>{\arrayrulecolor{DarkBackground}}--}
\SetRowColor{LightBackground}
\mymulticolumn{2}{x{8.4cm}}{\textasciicircum{}1\textasciicircum{} Klammern optional \newline \textasciicircum{}2\textasciicircum{}set() erzeugt eine leere Menge}  \tn 
\hhline{>{\arrayrulecolor{DarkBackground}}--}
\end{tabularx}
\par\addvspace{1.3em}

\begin{tabularx}{8.4cm}{X}
\SetRowColor{DarkBackground}
\mymulticolumn{1}{x{8.4cm}}{\bf\textcolor{white}{Funktionen}}  \tn
\SetRowColor{LightBackground}
\mymulticolumn{1}{x{8.4cm}}{def funktionsname(Var1, Var2=4): \newline \tab \#Anweisungen \newline \tab \#Anweisungen \newline \tab return result      \#optional} \tn 
\hhline{>{\arrayrulecolor{DarkBackground}}-}
\end{tabularx}
\par\addvspace{1.3em}

\begin{tabularx}{8.4cm}{X}
\SetRowColor{DarkBackground}
\mymulticolumn{1}{x{8.4cm}}{\bf\textcolor{white}{Selektionen}}  \tn
\SetRowColor{LightBackground}
\mymulticolumn{1}{x{8.4cm}}{if bedingung: \newline \tab  \#Anweisungen, falls bedingung erfüllt ist \newline {\emph{elif bedingung2:}} \newline     {\tab  \emph{\#Anweisungen}} \newline else: \newline \tab   \#Anweisungen} \tn 
\hhline{>{\arrayrulecolor{DarkBackground}}-}
\end{tabularx}
\par\addvspace{1.3em}

\begin{tabularx}{8.4cm}{x{1.748 cm} x{2.584 cm} x{3.268 cm} }
\SetRowColor{DarkBackground}
\mymulticolumn{3}{x{8.4cm}}{\bf\textcolor{white}{Bedingungen}}  \tn
% Row 0
\SetRowColor{LightBackground}
`\textless{}` & kleiner als & `a \textless{} 10` \tn 
% Row Count 1 (+ 1)
% Row 1
\SetRowColor{white}
`\textgreater{}` & gr{\"o}sser als & `b\textgreater{}4` \tn 
% Row Count 2 (+ 1)
% Row 2
\SetRowColor{LightBackground}
`==` & gleich & `c=='yes'` \tn 
% Row Count 3 (+ 1)
% Row 3
\SetRowColor{white}
`\textless{}=` & kleiner gleich & `d\textless{}=5` \tn 
% Row Count 5 (+ 2)
% Row 4
\SetRowColor{LightBackground}
`\textgreater{}=` & gr{\"o}sser gleich & `e\textless{}=7` \tn 
% Row Count 7 (+ 2)
% Row 5
\SetRowColor{white}
`!=` & ungleich & `g!='no'` \tn 
% Row Count 8 (+ 1)
% Row 6
\SetRowColor{LightBackground}
`'in'` & in & `'x' in 'mexico'` \tn 
% Row Count 9 (+ 1)
% Row 7
\SetRowColor{white}
`'not in'` & nicht in & `y not in 'mexico'` \tn 
% Row Count 11 (+ 2)
\hhline{>{\arrayrulecolor{DarkBackground}}---}
\end{tabularx}
\par\addvspace{1.3em}

\begin{tabularx}{8.4cm}{x{4.32 cm} x{3.68 cm} }
\SetRowColor{DarkBackground}
\mymulticolumn{2}{x{8.4cm}}{\bf\textcolor{white}{Zeichenketten (Strings)}}  \tn
% Row 0
\SetRowColor{LightBackground}
`str.lower()` & in Kleinbuchstaben umwandeln \tn 
% Row Count 2 (+ 2)
% Row 1
\SetRowColor{white}
`str.upper()` & in Grossbuchstaben umwandeln \tn 
% Row Count 4 (+ 2)
% Row 2
\SetRowColor{LightBackground}
`str.replace(old,new)` & old durch new ersetzen \tn 
% Row Count 6 (+ 2)
% Row 3
\SetRowColor{white}
`str.split()` & Teilt den String auf \tn 
% Row Count 8 (+ 2)
% Row 4
\SetRowColor{LightBackground}
`str{[}1:5{]}` & Zeichen 1-5 anzeigen \tn 
% Row Count 10 (+ 2)
% Row 5
\SetRowColor{white}
`list(str)` & erzeugt eine Buchstabenliste \tn 
% Row Count 12 (+ 2)
\hhline{>{\arrayrulecolor{DarkBackground}}--}
\end{tabularx}
\par\addvspace{1.3em}


\begin{tabularx}{8.4cm}{X}
\SetRowColor{DarkBackground}
\mymulticolumn{1}{x{8.4cm}}{\bf\textcolor{white}{Iterationen}}  \tn
\SetRowColor{LightBackground}
\mymulticolumn{1}{x{8.4cm}}{{\bf{ for-Schleifen }} \newline for item in list: \newline   \tab   \#Anweisungen für item \newline   \tab   \#Anweisungen für item` \newline \#Anweisungen nach der Schleife \newline  \newline for i in range(n): \newline  \tab  \#Anweisungen n mal Wiederholen \newline  \newline {\bf{while Schleife }} \newline while bedingung: \newline  \tab  \#Anweisungen} \tn 
\hhline{>{\arrayrulecolor{DarkBackground}}-}
\SetRowColor{LightBackground}
\mymulticolumn{1}{x{8.4cm}}{`range(n) = {[}0,1,2,3,...,n-1{]}` Liste mit den ersten n Zahlen \newline `break` beendet die Schleife. `continue` beendet den aktuellen Durchlauf}  \tn 
\hhline{>{\arrayrulecolor{DarkBackground}}-}
\end{tabularx}
\par\addvspace{1.3em}

\begin{tabularx}{8.4cm}{x{4.24 cm} x{3.76 cm} }
\SetRowColor{DarkBackground}
\mymulticolumn{2}{x{8.4cm}}{\bf\textcolor{white}{Arbeiten mit Listen}}  \tn
% Row 0
\SetRowColor{LightBackground}
`len(myList)` & L{\"a}nge von myList \tn 
% Row Count 1 (+ 1)
% Row 1
\SetRowColor{white}
`myList{[}i{]}` & i-tes Element der Liste \tn 
% Row Count 3 (+ 2)
% Row 2
\SetRowColor{LightBackground}
`myList{[}i:j{]}` & Ausschnitt von i bis j \tn 
% Row Count 5 (+ 2)
% Row 3
\SetRowColor{white}
`x in myList` & `True` wenn x in myList ist \tn 
% Row Count 7 (+ 2)
% Row 4
\SetRowColor{LightBackground}
`myList.append(x)` & x myList anh{\"a}ngen \tn 
% Row Count 8 (+ 1)
% Row 5
\SetRowColor{white}
\{\{nobreak\}\}`myList.insert(i,x)` & x vor der Stelle i einfügen \tn 
% Row Count 10 (+ 2)
% Row 6
\SetRowColor{LightBackground}
`myList{[}i{]}=x` & Element i ersetzen \tn 
% Row Count 11 (+ 1)
% Row 7
\SetRowColor{white}
`myList.remove(x)` & entfernt x aus myList \tn 
% Row Count 13 (+ 2)
% Row 8
\SetRowColor{LightBackground}
`myList.pop({[}i{]})` & entfernt das i-te Element \tn 
% Row Count 15 (+ 2)
\hhline{>{\arrayrulecolor{DarkBackground}}--}
\SetRowColor{LightBackground}
\mymulticolumn{2}{x{8.4cm}}{`myList={[}{]}`}  \tn 
\hhline{>{\arrayrulecolor{DarkBackground}}--}
\end{tabularx}
\par\addvspace{1.3em}

\begin{tabularx}{8.4cm}{x{2.56 cm} x{5.44 cm} }
\SetRowColor{DarkBackground}
\mymulticolumn{2}{x{8.4cm}}{\bf\textcolor{white}{Dictionarys}}  \tn
% Row 0
\SetRowColor{LightBackground}
`len(dict)` & L{\"a}nge von dict \tn 
% Row Count 1 (+ 1)
% Row 1
\SetRowColor{white}
`del dict{[}key{]}` & l{\"o}scht den Schlüssel key \tn 
% Row Count 3 (+ 2)
% Row 2
\SetRowColor{LightBackground}
\seqsplit{`dict.keys()`} & Liste von Schlüsseln \tn 
% Row Count 5 (+ 2)
% Row 3
\SetRowColor{white}
`key in dict` & Wahr wenn es den Schlüssel gibt \tn 
% Row Count 7 (+ 2)
\hhline{>{\arrayrulecolor{DarkBackground}}--}
\SetRowColor{LightBackground}
\mymulticolumn{2}{x{8.4cm}}{`dict = \{key: value, \}`}  \tn 
\hhline{>{\arrayrulecolor{DarkBackground}}--}
\end{tabularx}
\par\addvspace{1.3em}

\begin{tabularx}{8.4cm}{x{3.12 cm} x{4.88 cm} }
\SetRowColor{DarkBackground}
\mymulticolumn{2}{x{8.4cm}}{\bf\textcolor{white}{Dateien}}  \tn
% Row 0
\SetRowColor{LightBackground}
`open(file,mode)` & Datei {\"o}ffnen \tn 
% Row Count 2 (+ 2)
% Row 1
\SetRowColor{white}
`f.read()` & liest den ganzen File \tn 
% Row Count 3 (+ 1)
% Row 2
\SetRowColor{LightBackground}
`f.readline()` & liest eine Zeile \tn 
% Row Count 4 (+ 1)
% Row 3
\SetRowColor{white}
\seqsplit{`f.readlines()`} & liest alle Zeilen \tn 
% Row Count 5 (+ 1)
% Row 4
\SetRowColor{LightBackground}
`for line in f:` & Zeile für Zeile durchgehen \tn 
% Row Count 7 (+ 2)
% Row 5
\SetRowColor{white}
\seqsplit{`f.write(prompt)`} & schreibt in die Datei \tn 
% Row Count 9 (+ 2)
% Row 6
\SetRowColor{LightBackground}
`f.close()` & schliesst die Datei \tn 
% Row Count 10 (+ 1)
\hhline{>{\arrayrulecolor{DarkBackground}}--}
\SetRowColor{LightBackground}
\mymulticolumn{2}{x{8.4cm}}{mode: 'r' lesen, 'w' schreiben,'r+' lesen und schreiben, 'a' anh{\"a}ngen \newline readlines() erzeugt eine Liste von Zeilen}  \tn 
\hhline{>{\arrayrulecolor{DarkBackground}}--}
\end{tabularx}
\par\addvspace{1.3em}


% That's all folks
\end{multicols*}

\end{document}